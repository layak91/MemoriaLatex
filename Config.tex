


\documentclass[12pt,a4paper,spanish]{book}

\usepackage{amsmath,amssymb,amsfonts,mathrsfs}
\usepackage{graphicx}
\usepackage{anysize}
\usepackage[utf8]{inputenc}
%\usepackage[english]{babel}
%\usepackage[latin1]{inputenc}
\usepackage[spanish]{babel}  %Quita el comentario a esta línea y comenta la siguiente si quieres que los números romanos aparezcan en mayúsculas
%\usepackage[spanish, es-lcroman]{babel} 
\usepackage[bf]{caption}
\usepackage{subfigure}
\usepackage{rotating}
\usepackage{fontenc}
%\usepackage{mathaccent}
\usepackage{titlesec}
\usepackage{multirow} %Para poder centrar verticalmente el contenido de las celdas de una tabla
\usepackage{fancyhdr} %Para personalizar los encabezados y pies de página
\usepackage{acronym}  %Para expandir automáticamente los acrónimos
\usepackage[titletoc]{appendix} %Para que cambie el título y la forma de numerar los apéndices


%Define el formato de los encabezados y pies de página del índice y la lista de acrónimos
\fancypagestyle{plain}{ %Encabezado y pie para el índice y acrónimos
 \fancyhf{}  %Elimina encabezdo y pie (menos la línea del encabezado)
 \renewcommand{\headrulewidth}{0pt} %Elimina la línea del encabezado
 \fancyfoot[LE,RO]{\thepage}
}


\marginsize{2.5cm}{2.5cm}{2.5cm}{2.5cm}

\linespread{1.5}

% Esto es para redefinir los títulos que Latex pone por defecto
\addto\captionsspanish{\renewcommand{\contentsname}{Contenido}}
\renewcommand{\tablename}{Tabla}

%Esto es para redefinir el formato en que se presenta el título de los capítulos
\titleformat{\chapter}[hang]{\Huge\bfseries}{\fontsize{24}{60} Capítulo \thechapter{. }}{0pt}{\fontsize{24}{60}}


\newcommand{\reff}[1]{Figura \ref{#1}}
\newcommand{\refe}[1]{(\ref{#1})}
\renewcommand{\captionfont}{\small}

%Si el idioma es español las listas aparecen con un cuadradito.
%En inglés aparecen con un bullet...Esto redefine el bullet a cuadrado.
\renewcommand{\labelitemi}{\tiny{$^\blacksquare$}}

%\renewcommand{\figurename}{Fig.}

\hyphenation{mo-du-la-tion pa-ra-me-te-ri-zed res-pon-se cha-rac-ter par-ti-cu-la-ri-zing de-ve-lo-ped a-ve-ra-ging pro-ba-bi-li-ty me-cha-nism par-ti-cu-lar know-led-ge pro-ducts inter-operable to-po-lo-gy cha-rac-te-ri-zed stra-te-gy rea-li-za-da si-mu-la-cio-nes pro-pues-tas pro-pues-to cen-tra-li-za-das igua-la-do-res su-mi-nis-tro va-lo-ra-do in-ter-fe-ren-cia par-ti-cu-la-ri-ties mul-ti-plexing ins-ti-tu-te ca-rri-er maxi-mum carre-te-ras} %es un modo burdo de definir como romper una palabra

%\sloppy %permite que sea permisivo con las líneas sueltas


%%Para que las viñetas del segundo nivel aparezcan como a),b)...y no (a), (b)
\renewcommand{\theenumii}{\alph{enumii}}
\renewcommand{\labelenumii}{\textsf{\theenumii})}

%Para las url en la bibliografía
\usepackage{url}

%Para escribir codigo
\usepackage{listings}

\usepackage{color}
\usepackage{listings}
\lstset{ %
language=C++,                % choose the language of the code
basicstyle=\footnotesize,       % the size of the fonts that are used for the code
%numbers=left,                   % where to put the line-numbers
%numberstyle=\footnotesize,      % the size of the fonts that are used for the line-numbers
%stepnumber=1,                   % the step between two line-numbers. If it is 1 each line will be numbered
%numbersep=5pt,                  % how far the line-numbers are from the code
backgroundcolor=\color{white},  % choose the background color. You must add \usepackage{color}
showspaces=false,               % show spaces adding particular underscores
showstringspaces=false,         % underline spaces within strings
showtabs=false,                 % show tabs within strings adding particular underscores
%frame=single,           % adds a frame around the code
tabsize=2,          % sets default tabsize to 2 spaces
captionpos=b,           % sets the caption-position to bottom
breaklines=true,        % sets automatic line breaking
breakatwhitespace=true,    % sets if automatic breaks should only happen at whitespace
escapeinside={\%*}{*)}          % if you want to add a comment within your code
}




